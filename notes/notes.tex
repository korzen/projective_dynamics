\documentclass[11pt]{article}

\usepackage{amsmath,amsfonts}

\begin{document}

\title{Project Notes}
\author{Pavol Klacansky, Will Usher}
\maketitle

\section{Projective Dynamics Notes}

It sounds like they implement the projective implicit Euler solver? So we're minimizing

\begin{align}
	\frac{1}{2 h^2} || M^{1/2} (q - s_n) ||^2_F + \sum_i \left( \frac{w_i}{2} || A_i S_i q - B_i p_i ||^2_F
		+ \delta_{C_i} (p_i) \right)
\end{align}

with respect to $q$ and $p_i$ it sounds like.

\subsection{Local Solve}

A constraint is an elastic potential, with generic form

\begin{align}
	W(x, y) = d(x, y) + \delta_C(y)
\end{align}

$W(x, y)$ measures the potential energy going from state $x$ to $y$?

\begin{itemize}
	\item $x$ is our current state vector
	\item $y$ is the state we're projecting to. When minimizing the constraint this is
		where we're going too.
	\item $d(x, y)$ is some measure of the distance between states $x$ and $y$
	\item $\delta_C(y)$ acts as an indicator if we're on the constraint manifold $C$ or not,
		$\delta_C(y) = \infty$ if $y$ doesn't meet the constraint we're trying to satisfy (e.g. a
		spring at rest length) and $\delta_C(y) = 0$ if it does satisfy it.
\end{itemize}

The paper \cite{projective_dynamics} uses constraints specified like

\begin{align}
	W(q, p) = \frac{w}{2} || A q - B p ||^2_F + \delta_C(p)
\end{align}

$A$ and $B$ depend on the constraint being applied. To satisfy a constraint we project on to the manifold
with a constrained minimization with respect to $p$

\begin{align}
	\min_p \frac{w}{2} || A q - B p ||^2_F + \delta_C(p)
\end{align}

The delta function enforces that $p$ satisfies the constraint we're trying to resolve while the distance measure
$|| A q - B p ||^2_F$ works to find the closest valid state $p$ to our current state $q$.

The form mentioned in the local solve in the paper (5) in \cite{projective_dynamics} adds an extra
selection matrix $S_i$ to pick the vertices from $q$ that are involved in the constraint:

\begin{align}
	\min_{p_i} \frac{w_i}{2} || A_i S_i q - B_i p_i ||^2_F = \delta_{C_i}(p_i)
\end{align}

\subsubsection{Spring Constraints}

The spring constraint (and they mention this in the paper) works out to be exactly like \cite{fast_mass_spring},
though I'm a but unsure if for the constraint $w = k$. For a spring with stiffness $k$ and rest length $r$:

\begin{align}
	W(q, d) = \frac{k}{2} || q - d ||^2 + \delta_S(d)
\end{align}

Here $q$ is our current spring vector and $d$ is the vector we want to adjust it to, such that $||d|| = r$.
The constraint function $\delta_S(d)$ would be 0 only if $||d|| = r$, so we end up with exactly the same
constrained minimization as from \cite{fast_mass_spring}:

\begin{align}
	\min_{||d|| = r} \frac{k}{2} ||q - d||^2
\end{align}

I'm not sure here if the $k/2$ should be in here, would $k = w$ for a spring constraint? The minimization is
then equivalent to projecting the spring vector $q$ on to a sphere with the spring rest length as its radius,
alternatively:

\begin{align}
	d = r \frac{q}{||q||}
\end{align}

\subsubsection{Strain Constraints}

The strain constraint for a triangle is more confusing, its discrete form:

\begin{align}
	W(q, T) = \frac{w}{2} A || X_f X^{-1}_g - T ||^2_F + \delta_M (T)
\end{align}

\begin{align*}
	\delta_M (T) = \begin{cases}
		0 & T \in M \\
		\infty & T \not\in M
	\end{cases}
\end{align*}

$X_f = [q_j - q_i, q_k - q_i]$ is the triangle edges in the current config, $X_g$ is the triangle edges
in the rest configuration. So what is the meaning of $X_f X^{-1}_g$ ?

The matrix $T$ that we're minimizing over is $T \in M$ where $M$ is some set of matrices, for isotropic strain
limiting they mention it is the set of matrices with bounded singular values, $\sigma_{\min} < \sigma < \sigma_{\max}$,
so now does this mean we must compute the SVD to evaluate $\delta_M (T)$ since membership $M$ is defined by
the singular values of $T$?

And now we have two degrees of freedom, since $T \in \mathbb{R}^{2 \times 2}$ so do we need to iterate to find

\begin{align}
	\min_T = \frac{w}{2} A || X_f X^{-1}_g - T ||^2_F + \delta_M (T)
\end{align}

Or can we just clamp the singular values as Pavol mentions? To find $A$ and $B$ to bring back into the global
solve would we need to find them such that:

\begin{align}
	X_f X^{-1}_g &= A_i S_i q \\
	T &= B_i p_i
\end{align}

\subsubsection*{Open Questions for Ladislav/Tiantian:}
\begin{itemize}
	\item Does the constraint projection really need to be a full constrained optimization solve? It seems
		like it would be expensive to compute for general constraints then. e.g. for a simple spring constraint
		you can just find the same spring direction vector with length of the spring's rest length, but what
		about more complicated constraints?

	\item How does the distance measure for strain constraints $|| X_f X^{-1}_g - T ||^2_F$ relate back to the
		general form and the matrices used in the global step, $||Aq - Bp||^2_F$ ?

	\item How does the minimization of $T$ work for strain? Can we just clamp the singular values into the
		range and be done? Would that guarantee that $|| X_f X^{-1}_g - T ||^2_F$ is also minimized?

	\item Why the Frobenius norm? (maybe b/c it's rotation invariant?)
\end{itemize}

\subsection{Global Solve}

With $s_n = q_n + h v_n + h^2 M^{-1} f_{\text{ext}}$ (the inertia + external forces) we solve:

\begin{align}
	\left(\frac{M}{h^2} + \sum_i w_i S^T_i A^T_i A_i S_i \right) q =
		\frac{M}{h^2} s_n + \sum_i w_i S^T_i A^T_i B_i p_i
\end{align}

Where the $A_i$, $B_i$ are how the constraints effect the global system and $p_i$ incorporates the
result of the local constraint projections. Note that everything besides $q$ is either constant or
computed previously in the local step (the $p_i$'s) we're really just solving $A x = b$:

\begin{align*}
	A &= \left(\frac{M}{h^2} + \sum_i w_i S^T_i A^T_i A_i S_i \right) \\
	x &= q \\
	b &= \frac{M}{h^2} s_n + \sum_i w_i S^T_i A^T_i B_i p_i
\end{align*}

\subsubsection*{Open Questions for Ladislav/Tiantian:}
\begin{itemize}
	\item How do the $S_i$, $A_i$, $B_i$ incorporate the discrete constraint formulations from
		section $5$ in the paper \cite{projective_dynamics}?
	\item They mention choosing $A_i = B_i$ as differential coordinate matrices, what do they
		mean here? How does it help the solution?
\end{itemize}

\bibliography{notes_bib}{}
\bibliographystyle{plain}

\end{document}

